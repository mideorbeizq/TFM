%\chapter{Application of Artificial Neural Networks}
\label{chapter:application}

After describing the biological and mathematical background of the artificial neural networks (\charef{nn}) and the adquisition process of the physiological signals we are intended to use to predict migraine (\charef{setup}), this chapter exposes the work carried out during this project.


It comprises the adaptation process of the monitored signals in order to make them suitable for being the inputs of each type of neural network. 
As far as the signals are concerned, we highlight the fact that only 4 migraine episodes of the same patient were taken into account. 
The reason of this lies on the conclusion arrived in a previous work \cite{Irene:PFC:2014}: 
the prediction goal seems to be achieve only with a per-patient model. 
Then, as the monitoring period is short, few migraines are registered from each patient.
Therefore, for the training set we used truncate version of a maximum of two migraines. Then, the test phase is carried out with the entire four migraine episodes.


It is important to emphasize that,
in order to compare and get better results,
after build and evaluate each ANN topology,
its parameters are modified and the ANN perform is reanalysed. 
The set of the adjustable params includes:
the set of signals used as network inputs,
the windowing or truncation of the time series,
the amount of neurons in the network layers, 
the number of time steps of the TDLs if they exist,
the prediction horizon  
and even the way in which the target curve is detected.
In a first instance, the greatest set of physiological variables used to predict the migraine target curve includes heart rate (HR), electrodermal activity (EDA), skin temperature (TEMP) and oxigen saturation (SpO2).

Finally, since all networks weights are randomly initialised, 
in order to achive better networks performance, each ANN is initialised and evaluated several times. The best trained ANN will be chosen later based on the test results and not on the training results for avoiding any overfitting.