%\chapter{Introduction} 
\label{chapter:introduction}

The ageing population in Europe, not healthy lifestyles and the increase of chronic diseases are resulting in a general demand increment of health attention and resources \cite{pmid18843167}. This context and the fast development of information and communication technologies (ICT) give space for the e-Health concept: the ICT application to health field.

Among e-Health components are the electronic clinical history, telemedicine, e-Learning,  continuous education in ICT, standardization and interoperability and the mobile health (m-Health) \cite{OPSOMSEstrategia}. The last mentioned component, gives name to the medicine supported by mobile devices such as tablets and smartphones, monitoring artifacts, PDAs and other wireless gadgets \cite{kay2011mhealth}. The group interaction constitutes the named Wireless Body Sensor Networks (WBSN), a booming technology due to its wide application to the telemedicine and monitoring field. m-Health belongs to a wider stage, where a mass WBSN deployment supposes two aspects to take into account: the first one, the data generation susceptible of processing in data centers; the latter, the necessity of imposing strict power restrictions.

Nowadays, patients who are taking more advantage of m-Health are those in whose state the monitoring results critical; \ie, whose pathologies present anticipation to the problem as a requisite, so it is essential a continuous knowledge of patient physiological parameters. Patients with cardiovascular, diabetes or epilepsy problems are some instances.

Nevertheless, there are lots of health problems in which ICT will benefit. One of them is migraine pathology, a very frequent illness (it affects over 16\% and 8\% of world women and men, respectively) characterized by recurrent headache attacks. Even being episodic, migraine disease is a crippling pathology that involves life quality of whom suffer it, in personal, family, labor, and social fields \cite{SENmigsinaura}. It also supposes a big socioeconomic impact: European health services relate a cost of 1222\euro per patient per year~\cite{Linde:2012:CostMigraineEU}. Around $93\%$ of these costs are indirect costs, and they are due to reduced productivity at work ($66\%$ of indirect costs) and absenteeism ($33\%$).

Migraine usually develops in four steps: prodrome phase, which consists in subtle premonitory symptoms; aura phase, with more specific symptoms that are not always present; the strictly speaking pain phase; and postdrome or resolution and recovery phase \cite{SENmigaura, CUNcefmig}.

Patients usually take their medication when aura is identified (if it exists) or directly in the pain phase. However, this process does not allow the complete actuation of medicine, so headache persists generating the patient dissatisfaction with the treatment \cite{IBCinsatisfacc}.

The solution may be in the earlier treatment of migraine, preventing the pain phase by the pre-headache symptoms \cite{MigAgudaEarlyTreat}. Here is where the title of this project fits in, whose global aim is the prediction of migraine attack in order to make effective the medical treatment.

Within m-Health, the goal is expected to be reached monitoring hemodynamic variables of the patient, such as heart rate, corporal temperature, sweating and oxigen saturation with a WBSN. Even if there are not previous works that have analyzed the variation of these variables during the earlier stages of a migraine attack, the literature and the clinicians expect the sympathetic system of these patients to be affected and, therefore, the hemodynamic changes will occur. 

Analysing the evolution and influence of these variables in the different migraine phases, the idea is to develop an algorithm that predicts headaches enough in advance, and be able to notify the patient when to take the medicine. For the mentioned purpose, we propose some data mining approaches based on Artificial Neural Networks (ANNs) because of their admissible predictive performance. In fact, it is presently the most popular data modeling method used in the medical domain due to their ability of model highly nonlinear systems such as physiological records, where the correlation of the input parameters is not easily detectable.


The remainder of this document is as follows. Firstly, in \charef{objectives} and \charef{methodology}, we describe the main goals and the steps followed for carrying out this project. Secondary, \charef{stateofart} introduces the reader to the related work of the migraine disease, the WBSN within the e-Health and the algorithms normally applied to physiological signals. In \charef{nn}, we revise the theoretical background of biological and artificial ones, exposing some different types of the latter. Then, in \charef{setup}, it is described briefly the physiological signals involved in this project, as well as how they had been adquired. \charef{application} explains the process we have carried out in order to apply the ANNs to our scenario, exposes the results and analyses the influence of modifying the parameters that define each algorithm. The conclusions and the future work are discussed in \charef{conclusionsandfuturework} at the end.
