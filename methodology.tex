%\chapter{Methodology} 
\label{chapter:methodology}

Before this project was carried out, a wireless system of biomedical data acquisition was developed last year as part of the Final Degree Project of the author \cite{Irene:PFC:2014}. Straightaway, a few real patients were monitored. The collected data were the starting point of our study.

The main steps we have followed during the work agree the process described in \cite{banaee2013data} for any data mining approach:

\begin{enumerate}
	\item Preprocessing of the raw sensor data in order to remove noise, motion artifacts and sensor errors of the wearable sensor network.

	\item Feature extraction and selection to discover the main characteristics of a data set which are representative of our original data.

	\item Learning and modeling process using the features selected together with expert knowledge and metadata (stable parameters).

	\item Application of the model to new no-learnt data to perform the desired task of prediction.

\end{enumerate}

In particular, this project has focused on the last three steps, since first ones have been carried out by other members of the group. 

Accordingly, with the purpose of achieving our goals (\charef{objectives}), the step 3 included the study of several different algorithms, as well as the analysis of the influence of the parameters that define each one. In addition, within the step 4, we have compared the performance of the different algorithms implemented.

As far as materials are concerned, all the algorithms were developed in MatLab, making use of the Neural Network Toolbox for some of them. Additionally, \LaTeX was used for writing this document. 