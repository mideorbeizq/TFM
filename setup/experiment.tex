%\chapter{Experimental setup}
\label{sec:experiment}

This section summarises the relevant details of the clinical study, as
well as the main characteristics of the system used for the
acquisition of the input signals of the implemented algorithms. It
must be highlighted that the aforesaid system was not developed during
this project, but as part of the Final Degree Project of the
author \cite{Irene:PFC:2014}.


\subsection{The patients}
\label{subsec:thepatients}

Clinically, the experiment was accomplished with patients diagnosed
with migraine according to diagnostic criteria of the International
Headache Society (IHS). During the study, they took a portable
monitoring non-invasive smart system along, for a maximum of two
weeks, 24 hours a day.

Among the criteria to choose the patients, the following ones describe
the ideal profile of the volunteer:
\begin{itemize}
	\item Migraineurs aged 15 to 69 years.

	\item Migraineurs diagnosed with migraine by a headache
	specialist using ICHD-3 criteria \cite{pmid23771276}

	\item Migraineurs who have had, at least, one crisis per week
	during the last three months.

	\item Migraineurs that do not take drugs which can alter the
	variables we monitor

	\item Migraineurs with normal neurological exam

	\item Migraineurs who have signed an informed consent

	\item Migraineurs that have a minimum knowledge about using
	smartphones
\end{itemize}

Once a patient was accepted, the volunteer was asked for the
corresponding informed consent. We made an appointment with him and
the doctor in which we explained in detail the operation of the
system. Additionally, we provided the patient with a user manual of
the system in which our contacts could be found for consulting any
doubt.


\subsection{The system}
\label{subsec:thesystem}

The data acquisition was carried out with a WBSN integrated into a
broader multitier telemedicine system. The architecture implemented
involves two different sensing motes which communicate with an Android
platform via Bluetooth. The data are stored and transmitted through
the Internet to a \emph{Cloud} storage system where follow-up and
processing tasks are done.

The two used sensing motes were \emph{PLUX-Wireless Biosignals}
\cite{BIOPLUXws} and \emph{Nonin Onyx II} \cite{NONINws}. EDA, skin
temperature, ECG and EEG signals were acquired with the former and PPG
and SpO2 values were registered with the latter. The HR signal was
then calculated offline from the ECG signal by our peak-detection
algorithm. We highlight that the developed algorithms for migraine
prediction take as input a subset of four of these hemodynamic
signals: HR, EDA, skin temperature and SpO2.

The Bluetooth-enabled Android device receives the data wirelessly as
soon as each sensing mote takes a sample.  The sampling frequency for
the ECG and EEG sensors was 250 Hz, 1 Hz for the skin temperature and
EDA variables, 3 Hz for the SpO2 signal and 75 Hz for PPG.  This
introduced some synchronization issues to deal with.

The Android smartphone does not only receive and store the gathered
data, but also includes a form with questions related to every
headache. These questions include the time when the migraine phase
begins and ends, suffered symptoms and possible triggers of the
attack. The patients were asked to fill up the form as soon as a
headache has finished. Additionally, they were inquired about the
evolution of their pain intensity during the migraine crisis in order
to define a pain curve that constitutes the target signal of our
algorithms.
